% \section{Experimental prototype}
% Describe the specific use-case you worked with and the experimental prototype you have built.

% \subsection{Equations}
% This is an equation:
% \begin{equation}
% a+b=\gamma\label{eq}
% \end{equation}

% \begin{table}[htbp]
% \caption{This is a table}
% \begin{center}
% \begin{tabular}{|c|c|c|c|}
% \hline
% \textbf{Table}&\multicolumn{3}{|c|}{\textbf{Table Column Head}} \\
% \cline{2-4} 
% \textbf{Head} & \textbf{Table column subhead}& \textbf{Subhead}& \textbf{Subhead} \\
% \hline
% text & text & text & text \\
% \hline
% \end{tabular}
% \label{tab1}
% \end{center}
% \end{table}

% \begin{figure}[htbp]
% % \centerline{\includegraphics{fig1.png}}
% \caption{Example of a figure caption.}
% \label{fig}
% \end{figure}
% \section{Experimental prototype}
% Describe the specific use-case you worked with and the experimental prototype you have built.

% \subsection{Equations}
% This is an equation:
% \begin{equation}
% a+b=\gamma\label{eq}
% \end{equation}

% \begin{table}[htbp]
% \caption{This is a table}
% \begin{center}
% \begin{tabular}{|c|c|c|c|}
% \hline
% \textbf{Table}&\multicolumn{3}{|c|}{\textbf{Table Column Head}} \\
% \cline{2-4} 
% \textbf{Head} & \textbf{Table column subhead}& \textbf{Subhead}& \textbf{Subhead} \\
% \hline
% text & text & text & text \\
% \hline
% \end{tabular}
% \label{tab1}
% \end{center}
% \end{table}

% \begin{figure}[htbp]
% % \centerline{\includegraphics{fig1.png}}
% \caption{Example of a figure caption.}
% \label{fig}
% \end{figure}
\section{Experimental Prototype}

This section describes the implementation of the 3-user co-located VR training system using Meta Quest 3 headsets.

\subsection{Hardware Configuration}

The experimental testbed consists of:

\begin{itemize}
    \item \textbf{Headsets:} 3$\times$ Meta Quest 3
    \begin{itemize}
        \item Hand tracking: Optical tracking without controllers
    \end{itemize}
    
    \item \textbf{Network Infrastructure:}
    \begin{itemize}
        \item WiFi access point
        \item Dedicated router with QoS configuration
    \end{itemize}
    
    %\item \textbf{Physical Space:} 6m $\times$ 6m padded training area with Guardian boundaries
\end{itemize}

\subsection{Software Architecture}

The system is built on Unity 6000.0.62f1 with the following components:

\begin{itemize}
    \item \textbf{Meta XR SDK v81.0.0:} Core VR functionality, hand tracking, passthrough
    \item \textbf{Photon Fusion 1.1.0:} State synchronization networking framework
    \item \textbf{Universal Render Pipeline (URP):} Performance-optimized rendering
    \item \textbf{Custom Building Blocks:}
    \begin{itemize}
        \item Colocation Manager: Spatial alignment and calibration
        \item SSA Manager: Shared spatial anchors for persistent world
        \item Avatar synchronization with NetworkBehaviour
        \item MetricsLogger: Performance monitoring and CSV export
    \end{itemize}
\end{itemize}

\subsection{Calibration Protocol}

Spatial anchor-based alignment procedure:

\begin{enumerate}
    \item Host creates and shares spatial anchor using OVR Colocation Discovery API
    \item Guest devices discover and localize the shared anchor
    \item ColocationManager aligns each camera rig to anchor transform
    \item System calculates alignment error as Euclidean distance from anchor position
    \item Calibration error is continuously tracked and logged via MetricsLogger
\end{enumerate}

The system uses Meta's OVRSpatialAnchor API for alignment rather than manual hand tracking calibration, providing automatic spatial correspondence between co-located users.

\subsection{Demo Scenarios}

Three colocation demonstration scenes:

\begin{description}
    \item \textbf{ColocationDiscovery}: Demonstrates OVR Colocation Discovery API with shared spatial anchor creation, advertisement, discovery, and alignment between multiple headsets.
    
    \item \textbf{SpaceSharing}: Shows MRUK room sharing and colocation using spatial anchors to synchronize physical environment understanding across users.
    
    \item \textbf{SpatialAnchorsBasics}: Basic spatial anchor creation, persistence, and loading functionality.
\end{description}

\subsection{Data Collection}

The MetricsLogger system captures:

\begin{itemize}
    \item \textbf{Technical metrics:} Frame rate, frame time, network RTT via Photon Fusion, calibration error from ColocationManager (1Hz sampling)
    \item \textbf{Device metrics:} Battery level, battery temperature, estimated CPU usage, memory usage
    \item \textbf{Session metadata:} Scenario type, interface type, environmental conditions, duration
\end{itemize}

Metrics are logged to CSV files with session metadata in JSON format, stored in the device's persistent data path for later retrieval and analysis.
