\section{Approach}

This study investigates a co-located multi-user VR configuration using wireless headsets to address the gap between distributed VR collaboration systems and the requirements of shared physical training environments.

\subsection{System Design}

The experimental system consists of three Meta Quest 3 standalone headsets operating within a shared physical space, networked via WiFi infrastructure.

\subsubsection{Hardware Configuration}
\begin{itemize}
    \item \textbf{Headsets:} 3$\times$ Meta Quest 3
    \item \textbf{Network:} WiFi router with dedicated 5GHz channel for low-latency local multiplayer
    \item \textbf{Tracking:} Inside-out optical tracking with hand tracking capabilities
\end{itemize}

\subsubsection{Software Architecture}
Built on Unity 6000.0.62f1, the system integrates:
\begin{itemize}
    \item \textbf{Meta XR SDK v81.0.0:} Core VR and passthrough functionality
    \item \textbf{Photon Fusion 1.1.0:} State synchronization networking framework
    \item \textbf{Universal Render Pipeline (URP):} Performance-optimized rendering
    \item \textbf{MetricsLogger:} Custom component for 1Hz performance monitoring (FPS, RTT, Calibration Error, Thermals)
\end{itemize}

\subsubsection{Calibration Protocol}
To achieve precise co-location, the system employs an automated spatial anchor alignment procedure:
\begin{enumerate}
    \item Host creates and shares a spatial anchor via OVR Colocation Discovery API.
    \item Guest devices discover and localize the shared anchor.
    \item \textbf{ColocationManager} aligns each user's camera rig to the anchor's transform.
    \item Alignment error is continuously calculated as the Euclidean distance from the anchor position.
\end{enumerate}

\subsection{Research Methodology}

The prototype follows a technical demonstration approach to validate performance against evidence-based requirements:

\begin{itemize}
    \item \textbf{Network Latency:} Targeting $\leq 75$ms RTT (Van Damme et al.~\cite{VanDammeSam2024Iolo}) for effective collaboration.
    \item \textbf{Calibration Accuracy:} Targeting $<10$mm error (Reimer et al.~\cite{ReimerDennis2021CfSV}) for collision prevention.
    \item \textbf{Frame Rate:} Targeting stable 72Hz performance for comfort.
    \item \textbf{Session Duration:} Validating stability over 30--60 minute training sessions.
\end{itemize}

\subsection{Assumptions and Limitations}

\textbf{Hardware Constraints:} Mobile processing necessitates visual fidelity trade-offs to maintain target frame rates.
\textbf{Network Assumptions:} Reliable WiFi connectivity is assumed; performance depends on local infrastructure.
\textbf{Prototype Scope:} This is a technical demonstration prototype establishing baseline metrics for future user studies.