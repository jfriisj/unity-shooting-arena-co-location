\section{Approach}
% Describe your approach and solution. And describe how your solution is different (e.g. more robust, cheaper, user friendly, ecological, etc.)

This study investigates a co-located multi-user VR configuration using wireless headsets to address the gap between distributed VR collaboration systems and the requirements of shared physical training environments. Our approach translates evidence-based technical requirements from systematic literature review into an experimental testbed designed to validate performance benchmarks and assess training effectiveness.

\subsection{System Configuration}

The experimental system consists of three Meta Quest 3 standalone headsets operating within a shared physical space, networked via WiFi infrastructure. This configuration enables investigation of co-located collaboration while meeting the wireless operation requirement that correlates with improved presence in prior paramedic training studies \cite{SchildJonas2018E—Ev} and the low-latency threshold ($\leq 75$ms RTT) established for effective collaborative VR \cite{VanDammeSam2024Iolo}.

\textbf{System Components:}
\begin{itemize}
    \item \textbf{Display Hardware:} 3$\times$ Meta Quest 3 headsets (2064$\times$2208 per eye, 90Hz refresh rate, inside-out tracking)
    \item \textbf{Network Infrastructure:} WiFi networking for local multiplayer
    \item \textbf{Software Platform:} Unity 6000.0.62f1 with Meta XR SDK v81.0.0, Photon Fusion 1.1.0 networking
    \item \textbf{Colocation System:} OVR Colocation Discovery API with shared spatial anchors
    \item \textbf{Safety Measures:} Guardian boundary system with real-time tracking monitoring
\end{itemize}

\subsection{Research Contributions}

This research addresses four gaps in current multi-user VR literature:

\subsubsection{Co-Located Configuration Demonstration}
Co-Located Configuration Demonstration: While prior research has examined distributed multi-user VR \cite{VanDammeSam2024Iolo} and 2-user co-located systems \cite{ReimerDennis2021CfSV}, this prototype demonstrates 3-user co-located configuration using Quest 3 wireless headsets withautomated spatial anchor alignment, investigating technical feasibility and baseline performance characteristics.

\subsubsection{Consumer Hardware Performance Assessment}

Existing co-located VR research primarily employs PC-tethered systems \cite{SchildJonas2018E—Ev} or older standalone hardware \cite{ReimerDennis2021CfSV}. This prototype demonstrates current-generation consumer standalone headsets (Quest 3) with automated spatial anchor colocation, collecting baseline performance metrics (calibration accuracy, network latency, frame rate) to assess feasibility for co-located applications.

\subsubsection{Evidence-Based Design Methodology}

Our system design integrates validated technical requirements from systematic review:

\begin{itemize}
    \item \textbf{Network latency:} Photon Fusion networking targeting Van Damme et al.'s $\leq 75$ms threshold for good collaborative QoE~\cite{VanDammeSam2024Iolo}, with continuous RTT monitoring via metrics logger implementation
    
    \item \textbf{Calibration approach:} Spatial anchor-based alignment using Meta's OVR Colocation Discovery API for automatic spatial correspondence, measuring alignment accuracy from anchor position
    
    \item \textbf{Visual consistency:} Symmetric rendering across all headsets using shared spatial anchors per Weiss et al.'s findings on asymmetry-induced collaboration degradation~\cite{WeissYannick2025ItEo}
\end{itemize}

This methodology enables systematic comparison between evidence-based requirements and measured system performance.

\subsubsection{Wireless Operation Characteristics}

Building on Schild et al.'s correlational findings between wireless operation and improved usability, this prototype demonstrates wireless co-located system capabilities. The Quest 3 configuration eliminates cable management constraints inherent in tethered systems.



\subsection{Implementation Approach}

The prototype follows a technical demonstration approach combining performance metrics collection with co-located interaction capabilities:

\subsubsection{Technical Validation (RQ1, RQ2)}

Network latency, calibration accuracy, and frame rate are continuously logged via the MetricsLogger implementation. Calibration validation uses spatial anchor-based alignment through OVR Colocation Discovery API, with alignment error measured as Euclidean distance from anchor position. Network performance monitoring captures RTT from Photon Fusion at 1Hz sampling rate. Frame rate and frame time metrics are recorded throughout sessions.

\subsubsection{Collaboration Assessment (RQ3, RQ4)}

Demo scenarios demonstrate co-located interaction capabilities using shared spatial anchors and networked object synchronization. The system enables investigation of spatial coordination and collaboration quality in co-located VR contexts.

\subsubsection{Performance Monitoring}

Technical performance metrics are collected through the MetricsLogger system:
\begin{itemize}
    \item \textbf{Frame Rate:} FPS and frame time tracking
    \item \textbf{Calibration:} Spatial anchor alignment error from ColocationManager
    \item \textbf{Network Performance:} RTT monitoring via Photon Fusion
    \item \textbf{Device Metrics:} Battery level, battery temperature, estimated CPU usage, memory usage
\end{itemize}

These metrics enable assessment of consumer wireless hardware performance during co-located VR sessions, providing baseline data for future research.

\subsubsection{Implementation Scope}

The prototype includes demonstration scenes showcasing colocation capabilities (ColocationDiscovery, SpaceSharing, SpatialAnchorsBasics). Performance metrics are logged via MetricsLogger to establish baseline characteristics of multi-user co-located VR on consumer hardware.

\subsection{Assumptions and Limitations}

\textbf{Hardware Constraints:} Quest 3's mobile processing (Snapdragon XR2 Gen 2) necessitates visual fidelity trade-offs to maintain target frame rates. Performance-optimized rendering prioritizes frame rate stability.

\textbf{Network Assumptions:} The system assumes reliable WiFi connectivity for local multiplayer. Network performance (latency, stability) depends on local infrastructure quality and environmental factors. Literature suggests WiFi 6E with dedicated 6GHz spectrum could achieve $<30$ms latency~\cite{VanDammeSam2024Iolo}, though our implementation uses available WiFi infrastructure.

\textbf{Calibration Approach:} The prototype uses Meta's automated spatial anchor alignment rather than manual hand tracking-based calibration explored in literature~\cite{ReimerDennis2021CfSV}. While manual methods can achieve $<10$mm accuracy, spatial anchors provide automatic correspondence suitable for demonstration purposes. Drift monitoring and re-alignment capabilities exist but require extended validation.

\textbf{Prototype Scope:} This is a technical demonstration prototype, not a complete research study. Full validation would require: (1) formal user studies with IRB approval, (2) structured training scenarios with task metrics, (3) comparative interface studies (e.g., WIM vs. baseline), and (4) extended session testing (30-60 minutes) with multiple participant groups. Current implementation provides technical foundation and baseline metrics for such future research.