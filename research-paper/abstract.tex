% \begin{abstract}
% Summarize your work in 100-150 words.
% \end{abstract}
\begin{abstract}
Co-located multi-user virtual reality (VR) training enables participants to share physical space while collaborating in virtual environments, yet its technical feasibility using consumer hardware remains under-validated. This study evaluates a 3-user co-located system using Meta Quest 3 wireless headsets against evidence-based requirements for professional training. We assessed key performance indicators including network latency, spatial calibration precision, and long-term stability under thermal load. The system demonstrated the capability to maintain safety-critical standards throughout extended sessions, achieving precise user alignment and stable frame rates. Our findings validate that current-generation consumer wireless VR hardware can effectively support professional co-located training scenarios at a fraction of the cost of traditional enterprise solutions, significantly lowering the barrier to entry for collaborative immersive simulation.
\end{abstract}