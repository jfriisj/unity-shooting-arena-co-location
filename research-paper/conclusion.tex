\section{Conclusion}

This study validates consumer-grade wireless VR headsets for professional co-located training applications, demonstrating that Meta Quest 3 systems achieve enterprise-grade performance benchmarks at 15--20\% of traditional costs.

\subsection{Key Findings}

\textbf{Technical Validation (RQ1):} The 3-user co-located system exceeded all evidence-based requirements. Network latency (30.8ms) remained 59\% below Van Damme et al.'s threshold~\cite{VanDammeSam2024Iolo}, initial calibration accuracy (4.1mm) surpassed Reimer et al.'s safety target~\cite{ReimerDennis2021CfSV} by 59\%, and frame rates maintained near-target performance (89.7fps) throughout 45-minute sessions.

\textbf{Collaboration Enhancement (RQ2/RQ3):} WIM spatial awareness interfaces demonstrated significant benefits: 27.5\% faster task completion ($p=0.015$), 65.9\% reduction in coordination errors ($p=0.002$), and 33.3 percentage point increase in task success rates. These findings extend Chen et al.'s work~\cite{ChenLei2024Eoep} to 3-user co-located configurations, confirming WIM effectiveness scales with scenario complexity.

\textbf{Long-Term Stability (RQ4):} Thermal analysis revealed predictable performance degradation patterns strongly correlated with device temperature ($r=-0.94$ for FPS, $r=0.96$ for calibration). The system maintains enterprise-grade metrics for 45 minutes, after which mid-session recalibration protocols restore accuracy. This establishes practical session duration guidelines for safety-critical training.

\subsection{Practical Implications}

\textbf{Cost-Effectiveness:} At approximately \$1,500 total hardware cost (3$\times$ Quest 3 headsets) plus consumer networking equipment, this system delivers 98\% of enterprise VR room capabilities at 15--20\% of typical costs (\$50,000+ for solutions like Virtualware VIROO). This democratizes access to high-fidelity collaborative VR training for smaller organizations, educational institutions, and distributed training facilities.

\textbf{Deployment Flexibility:} Wireless operation eliminates cable management complexity that plagued 66.7\% of users in Schild et al.'s study~\cite{SchildJonas2018E—Ev}. The portable configuration enables training in authentic physical environments (e.g., actual emergency response sites, maintenance facilities) rather than dedicated VR rooms, supporting context-specific skill development.

\textbf{Safety Validation:} Achieving $<5$mm calibration accuracy during optimal performance phases provides substantial safety margins for collision prevention in co-located training. Thermal monitoring and automated recalibration protocols ensure continuous safety compliance even during extended use.

\subsection{Limitations}

\textbf{Session Duration:} While 45-minute windows accommodate most training modules, longer sessions require mid-session recalibration. Future work should investigate active thermal management (e.g., ventilation systems, duty cycling) to extend optimal performance duration.

\textbf{User Count Scalability:} This study validates 3-user configurations. Scaling to 4+ users may introduce tracking interference and network congestion requiring additional infrastructure. The evidence-based methodology established here provides a framework for systematic scalability assessment.

\textbf{Subjective Evaluation Scope:} The current study focuses on technical validation and objective performance metrics. Comprehensive user experience assessment (presence, usability, cybersickness, learning outcomes) requires larger-scale studies with institutional ethics approval and remains as critical future work.

\textbf{Scenario Generalization:} Testing employed three representative training scenarios. Domain-specific validation (emergency medical services, industrial maintenance, crisis response) is necessary before operational deployment in safety-critical contexts.

\subsection{Future Work}

\textbf{Thermal Optimization:} Investigate performance-balanced rendering techniques, dynamic level-of-detail systems, and computational offloading strategies to reduce thermal load while maintaining visual fidelity sufficient for training effectiveness.

\textbf{Advanced Calibration:} Develop computer vision-based continuous recalibration using shared spatial features (e.g., furniture edges, room geometry) detected through passthrough cameras, eliminating manual recalibration interruptions.

\textbf{Scalability Studies:} Systematically evaluate 4-, 5-, and 6-user configurations to determine practical upper limits for consumer wireless infrastructure and identify network architecture optimizations (e.g., dedicated server nodes, hybrid peer-to-peer/client-server topologies).

\textbf{Learning Effectiveness:} Conduct controlled studies comparing training outcomes between co-located VR, remote VR, and traditional methods. Measure skill retention, transfer to real-world performance, and cost-effectiveness from pedagogical perspective.

\textbf{Domain-Specific Validation:} Partner with professional training organizations to deploy and validate the system in authentic use cases: paramedic teams practicing emergency procedures, maintenance crews rehearsing complex repairs, crisis response coordination drills.

\subsection{Concluding Remarks}

This research demonstrates that current-generation consumer VR hardware has matured to the point of viability for professional training applications. By systematically validating performance against evidence-based benchmarks derived from literature review, we provide a methodological framework for future co-located multi-user VR system development.

The convergence of affordable standalone headsets, robust wireless networking, and sophisticated spatial tracking enables a new paradigm: portable, collaborative VR training systems deployable anywhere teams gather. This shifts VR from expensive fixed installations to flexible tools accessible to organizations of all sizes, democratizing access to immersive learning technologies previously reserved for well-funded institutions.

As consumer VR technology continues advancing, the evidence-based evaluation methodology established here will enable rapid assessment of new hardware generations, ensuring training systems remain current with state-of-the-art capabilities while maintaining safety and effectiveness standards required for professional applications.
