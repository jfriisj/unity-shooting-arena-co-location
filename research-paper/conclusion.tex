\section{Conclusion}

This study validates consumer-grade wireless VR headsets for professional co-located training applications, demonstrating that Meta Quest 3 systems achieve enterprise-grade performance benchmarks at 15--20\% of traditional costs.

\subsection{Key Findings}

\textbf{Technical Validation (RQ1):} The 3-user co-located system met or exceeded all evidence-based requirements. Network latency ($67.4 \pm 268.6$ms) remained within Van Damme et al.'s threshold~\cite{VanDammeSam2024Iolo} for good QoE, while automated calibration accuracy ($0.28 \pm 0.67$mm) significantly surpassed Reimer et al.'s safety target~\cite{ReimerDennis2021CfSV}. Frame rates stabilized at the native 72Hz target ($72.3$fps) throughout extended sessions.

\textbf{Long-Term Stability (RQ4):} Thermal analysis revealed predictable performance patterns. Despite temperature increases of up to $+14.2^{\circ}$C, the system maintained stable performance for sessions exceeding 90 minutes, validating the hardware's capability for extended training durations without critical thermal throttling.

\subsection{Practical Implications}

\textbf{Cost-Effectiveness:} At approximately \$1,500 total hardware cost (3$\times$ Quest 3 headsets) plus consumer networking equipment, this system delivers 98\% of enterprise VR room capabilities at 15--20\% of typical costs. This democratizes access to high-fidelity collaborative VR training.

\textbf{Deployment Flexibility:} Wireless operation eliminates cable management complexity. The portable configuration enables training in authentic physical environments rather than dedicated VR rooms.

\textbf{Safety Validation:} Achieving sub-millimeter calibration accuracy provides substantial safety margins for collision prevention in co-located training.

\subsection{Limitations}

\textbf{Network Stability:} While mean latency was acceptable, occasional spikes indicate the need for robust error handling in production environments.
\textbf{User Count Scalability:} This study validates 3-user configurations. Scaling to 4+ users may introduce network congestion requiring additional infrastructure.

\subsection{Future Work}

\textbf{Scalability Studies:} Systematically evaluate 4-, 5-, and 6-user configurations to determine practical upper limits.
\textbf{Learning Effectiveness:} Conduct controlled studies comparing training outcomes between co-located VR, remote VR, and traditional methods.

\subsection{Concluding Remarks}

This research demonstrates that current-generation consumer VR hardware has matured to the point of viability for professional training applications. By systematically validating performance against evidence-based benchmarks, we provide a methodological framework for future co-located multi-user VR system development. The convergence of affordable standalone headsets, robust wireless networking, and sophisticated spatial tracking enables a new paradigm of portable, collaborative VR training.
