\section{Introduction}

Virtual Reality (VR) training systems have demonstrated significant potential for professional education, particularly in safety-critical domains such as emergency medical services \cite{SchildJonas2018E—Ev}, maintenance operations \cite{HeinonenHanna2022EtBo}, and crisis response \cite{SharmaSharad2025IASR}.
However, most existing multi-user VR solutions address scenarios where users connect from geographically distributed locations \cite{VanDammeSam2024Iolo}, leaving a critical gap in systems designed for colocated training where multiple users share the same physical space. This distinction is not merely technical—co-located training enables real-world teamwork dynamics, immediate physical assistance between trainees, and authentic spatial coordination that remote systems cannot replicate. Standalone wireless VR headsets, particularly the Meta Quest 3, present unprecedented opportunities for co-located multi-user training, eliminating cable-related usability issues that caused 66.7\% of users to rate tethered systems poorly \cite{SchildJonas2018E—Ev}. However, co-located systems introduce unique challenges achieving calibration accuracy to prevent user collisions \cite{ReimerDennis2021CfSV}, maintaining low network latency \cite{VanDammeSam2024Iolo}, providing spatial awareness interfaces \cite{ChenLei2024Eoep}, and ensuring visual consistency across headsets \cite{WeissYannick2025ItEo}. While enterprise VR rooms (e.g., Virtualware’s VIROO) demonstrate commercial viability, their high costs (\$50,000+) limit accessibility. This work addresses the need for an affordable, portable co-located VR system using Meta Quest 3 headsets ($\approx$\$300 each), delivering enterprise training capabilities at 15-20\% of traditional costs while solving co-location challenges: collision prevention \cite{ReimerDennis2021CfSV}, spatial awareness \cite{ChenLei2024Eoep}, and millimeter-accurate avatar alignment.


\subsection{Motivation and Significance}

Professional training in safety-critical domains demands realistic, repeatable, and scalable practice. Traditional highfidelity training rigs are costly and logistically heavy, limiting frequent practice and team-based drills \cite{SharmaSharad2025IASR}. Consumer wireless headsets (e.g., Meta Quest 3) promise orders-of-magnitude cost reduction, but require validation against evidencebased technical benchmarks: precise calibration (\textless10mm) for collision avoidance \cite{ReimerDennis2021CfSV}, low end-to-end latency to preserve shared action timing \cite{VanDammeSam2024Iolo}, and spatial awareness interfaces to support multi-user coordination \cite{ChenLei2024Eoep}. Visual consistency across participants is also critical—mismatches degrade collaborative performance \cite{WeissYannick2025ItEo}. This work addresses the technical foundation for such validation by implementing a 3-user co-located prototype that demonstrates automated spatial anchor colocation, performance monitoring infrastructure, and networked synchronization capabilities. The system provides an open-source reference implementation and establishes metrics collection framework for future empirical studies to measure learning, safety, and usability outcomes with actual participants.

\subsection{Research Questions}
This prototype implementation addresses key technical challenges identified in co-located multi-user VR literature. The system design targets integration of validated requirements from systematic review to create a foundation for future empirical research.

Research Questions:
\begin{itemize}
    \item \textbf{RQ1}: How can automated spatial colocation be implemented for 3-user configurations using Meta's OVR Colocation Discovery API with spatial anchor-based alignment?
    \item \textbf{RQ2}: What performance monitoring infrastructure is required to enable continuous metrics collection (network latency, frame rate, calibration accuracy, device thermals) for future validation studies?
    \item \textbf{RQ3}: How effectively can networked object synchronization and state management be achieved across co-located headsets using Photon Fusion networking framework?
    \item \textbf{RQ4}: What architectural patterns enable reproducible implementation addressing literature-identified challenges: collision prevention through calibration, low-latency local networking, and symmetric rendering?
    \item \textbf{RQ5}: What are the technical feasibility boundaries and limitations of consumer wireless VR hardware (Quest 3) for professional training applications requiring safety-critical performance?
\end{itemize}
